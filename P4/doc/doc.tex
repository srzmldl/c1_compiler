% Created 2015-10-31 Sat 01:26
\documentclass[11pt]{article}
\usepackage[utf8]{inputenc}
\usepackage[T1]{fontenc}
\usepackage{fixltx2e}
\usepackage{graphicx}
\usepackage{longtable}
\usepackage{float}
\usepackage{wrapfig}
\usepackage{rotating}
\usepackage[normalem]{ulem}
\usepackage{amsmath}
\usepackage{textcomp}
\usepackage{marvosym}
\usepackage{wasysym}
\usepackage{amssymb}
\usepackage{hyperref}
\tolerance=1000
\author{srzmldl}
\date{\today}
\title{doc}
\hypersetup{
  pdfkeywords={},
  pdfsubject={},
  pdfcreator={Emacs 24.5.1 (Org mode 8.2.10)}}
\begin{document}

\maketitle
\tableofcontents

\section{学习Bison}
\label{sec-1}
\subsection{回答问题}
\label{sec-1-1}
\subsubsection{makefile中expr编译任务}
\label{sec-1-1-1}
expr,expr1,exprL, exprL1的命令如下:
\begin{verbatim}
$(YACC) -b $@ -o $(SRC)/$@.tab.c $(CONF)/$@.y
$(LEX) -o$(SRC)/$@.lex.c $(CONF)/expr.lex
$(CC) -o $(BIN)/$@ $(SRC)/$@.lex.c $(SRC)/$@.tab.c -ll -lm
\end{verbatim}
可以尝试用`make expr`得到具体执行命令
\begin{verbatim}
bison -d -y -b expr -o src/expr.tab.c config/expr.y
flex -i -I  -osrc/expr.lex.c config/expr.lex
gcc -g -Iinclude -o bin/expr src/expr.lex.c src/expr.tab.c -ll -lm
\end{verbatim}
然后man一下bison,flex,gcc的
\begin{enumerate}
\item 第一条bison命令
\label{sec-1-1-1-1}
作用就是用bison将expr.y将类yacc翻译成c语言文件expr.tab.c
\begin{itemize}
\item -d 产生头文件,但是不能指定文件(for POSIX Yacc)
\item -y 模拟POSIX Yacc
\item -b 指定输出文件前缀
\item -o 输出到文件
\end{itemize}
\item 第二条flex指令
\label{sec-1-1-1-2}
作用就是用flex将expr.lex中的正规式以及匹配动作等翻译到expr.lex.c中去
\begin{itemize}
\item -i ignore case in patterns
\item -I 生成交互式的 scanner (与-B相对)
\item -o 指定输出文件名
\end{itemize}
\item 第三条gcc指令
\label{sec-1-1-1-3}
作用是用gcc编译expr.lex.c和expr.tab.c,生成可执行文件expr.
\begin{itemize}
\item -o 指定输出的可执行文件名称
\item -g 生成调试信息,方便gdb等调试
\item -I 指定头文件查找目录
\item -lm link math库
\item -ll link了某个库吧.我木有用过这个.
\end{itemize}
\end{enumerate}
\subsubsection{描述Yacc输入文法规范文件的格式,消化语法制导的翻译方案}
\label{sec-1-1-2}
\begin{enumerate}
\item 综述
\label{sec-1-1-2-1}
yacc语法规范由三部份组成
声明
\%\%
翻译规则
\%\%
C语言编写的支持例程
\item 声明
\label{sec-1-1-2-2}
声明部份由两节,均为可选. 
\begin{itemize}
\item 第一节'\%\{'和'\%\}'包起来,就是普通C语言的声明,如expr1.y和expr.y中都include了stdio.h.
\item 第二节声明一些文法终结符,比如NUMBER,EOL等.
\end{itemize}
\item 翻译规则
\label{sec-1-1-2-3}
每条规则由一个文法产生式以及和它联系的语义动作组成.没有加引号的字母数字串,若未声明成token,则式非终结符;加引号的单个字符,看城这个字符代表的记号.右部各个选择机器语义动作之间竖线隔开,最后一个选择后面用分号,表示产生式集合的结束.第一个左部非终结符式开始符号.

右部大括号括起来的语义动作终\$\$表示引用左部非终结符的属性值,\$i表示引用右部第i个文法符号的属性值.

例如exp-->exp PLUS exp \{$$ = $1 + $3;}就是把右部的两个exp值相加,结果赋值给左部的exp. 应该注意到语义动作是可以省略的.右部只有一个文法符号时,默认动作是{$$=\$1\}.
\item C语言例程
\label{sec-1-1-2-4}
\begin{itemize}
\item 词法分析器yylex()返回记号,这里由flex产生
\item yyerror() 输出错误信息
\end{itemize}

\item 语法制导的翻译方案
\label{sec-1-1-2-5}
expr.y为简单的表达式二义文法,通过指定优先级消除二义性. 包括加减乘除指数,负号,小括号.按照顺序指定了优先级,在规约规约冲突时选择先出现的表达式. 取负号的\%prec标签强制优先级和结合性MINUS一样. 这个翻译反感根据对应token进行相应加减乘除操作即可,十分简单.

expr1.y是非二义文法,翻译方案也是简单的在规约时根据规约式确定运算. 引入了fact和term消除二义性.
\end{enumerate}
\subsubsection{expr.tab.c和expr1.tab.c的异同; expr.y和expr1.y对expr.tab.c和expr1.tab.c影响}
\label{sec-1-1-3}
\begin{enumerate}
\item 异同
\label{sec-1-1-3-1}
\begin{itemize}
\item 二者最后实现的功能完全一致,只是expr.y主要利用yacc对优先级和左右结合的处理,以及\%prec标签比较暴力地解决二义性;但是expr1.y通过修改文法来消除二义性.最终殊途同归,都正确的计算表达式的值
\item 有一大段共享代码,主要差别在一些常量和宏定义,以及状态机,以及动作.
\item 修改文法之后expr1的YYNNTS变成了6个,用来消除二义性;语法规则也变成15条,变得更多;YYNSTATES状态数expr1为25,expr为23,状态数更多;YYLAST为YYTABLE的最后一个索引,expr为42,expr1为25;终结符这些当然是一样的,都是12个.
\item 除了上面这点描述状态机相关的变量或常量的差别之外,还有1220行左右的case中的动作根据文法中的规定而异
\item 除了上述两个差别,以及一些line中中行号的改编,其他部份基本一致
\end{itemize}
\item 影响
\label{sec-1-1-3-2}
\begin{itemize}
\item 由异同可见主要代码结构没变,对状态机的利用方式没变.
\item 因为文法不同,构造的状态机自然不同.那么就会影响yytokennum,yypact等和状态机有关的变量以及YYLAST, YYNTOKENS等这类和状态机有关的宏定义;expr.tab.h中的一大堆define也因文法而变
\item 1220多行附近yyreduce标签里yyn的switch里的动作也因翻译规则而异
\end{itemize}
\end{enumerate}
\subsubsection{了解L-asgn分析器的构成方法,简述.tab.c和.lex.c文件的结构}
\label{sec-1-1-4}
asgn和asgn1基本一样,只是asgn1中用'op'代替了所有中缀运算符.所以这里只讲解更为复杂的asgn1.
\begin{enumerate}
\item L-asgn分析器的构成方法
\label{sec-1-1-4-1}
\begin{itemize}
\item 先把文法写到asgn1.y中,通过bison翻译成C语言文件asgn1.tab.c
\item 写好asgn1.lex词法分析,通过flex翻译成C语言文件asgn1.lex.c
\item gcc编译成L-asgn分析器.输入是L-asgn语法文件,输出结果.
\end{itemize}
\item asgn1.tab.c结构
\label{sec-1-1-4-2}
\begin{itemize}
\item 大量宏定义,定义了Bison版本信息,终结符的枚举值,描述状态机需要的状态,转移边以及相应动作. 如 yystos
\item 然后是打印位置,符号,数值等信息的函数.如YY$_{\text{SYMBOL}}$$_{\text{PRINT}}$
\item 接下来是一些错误处理函数,如yysyntax$_{\text{error}}$
\item 然后空间释放函数yydestruct
\item 然后是转化函数 yyparse,其中包括规约,各种错误处理,接收等程序段,可以剪刀很多用来goto的标签,从命名上可以看出意义.如
\end{itemize}
yyacceptlab,yyreduce等     
\begin{itemize}
\item 最后是错误信息输出函数以及main函数,这是asgn1.y中定义的函数.
\end{itemize}
\item asgn1.lex.c结构
\label{sec-1-1-4-3}
\begin{itemize}
\item 大量宏定义,定义了flex版本信息,一些常用的小函数,常量等.
\item 定义正规式匹配的状态机相关的信息,如状态,转移矩阵等
\item 然后是scanner函数,包含了匹配段,动作段等,从goto标签yy$_{\text{match}}$,do$_{\text{actio等可以区分各段}}$
\item 然后是一些状态机上要用的函数的实现,比如找前趋状态的函数yy$_{\text{get}}$$_{\text{previous}}$$_{\text{state等}}$
\item 接下来是缓冲区各种操作函数的实现,如create,delete等操作.
\item 然后是一些yyset$_{\text{in}}$,yyset$_{\text{out之类接口的实现}}$
\item 最后是空间操作,如yyalloc,yyfree等
\end{itemize}
\end{enumerate}
\subsection{Bison理解}
\label{sec-1-2}
\subsubsection{规范格式}
\label{sec-1-2-1}
见问题2.
\subsubsection{与flex的协作}
\label{sec-1-2-2}
\begin{itemize}
\item lex文件中的头文件不再是自己显式定义,而是include "expr.tab.h",由bison生成,flex使用.该头文件包含了各种记号的定义,yylval的声明等
\item 词法分析器的第三部份函数被删除.事实上语法分析器会调用词法分析器. 可以在expr.tab.c看到很多的yylex()调用词法分析器取token
\item 我们在gcc编译的时候把expr.tab.c和expr.lex.c链接到了一起
\end{itemize}
\subsubsection{接口}
\label{sec-1-2-3}
\begin{itemize}
\item main函数调用yyparse()对输入文本进行分析,看是否符合语言规范
\item 语法分析器会不断调用yylex()通过词法分析器取token,并且二者共享符号表.
\end{itemize}
\subsubsection{文法对状态空间影响}
\label{sec-1-2-4}
通过上一部份问题中expr,expr1的比较,二者虽然实现的功能完全一致,但是文法不同.一个是二义文法,通过优先级与定义左右结合来消除冲突;另一个将文法改成无二义文法. 二者状态数不一样,转移表也发生了很大的变化. 由于产生式发生变化,规约动作等也出现了差异. 所以即使接受同样的语言规范,不同文法的状态空间可以由很大差异.
\section{C1程序的分析器}
\label{sec-2}
\subsection{reduce/reduce冲突}
\label{sec-2-1}
\subsubsection{单目运算符和双目运算符冲突}
\label{sec-2-1-1}
一开始直接老师给的文法,实际上是需要把单目运算符单独放到产生式里边就行了.
\begin{itemize}
\item exp --> '+' Exp
\item exp --> '-' Exp
\item 用\%prec标签规定下优先级
\end{itemize}
\subsection{shift/reduce冲突}
\label{sec-2-2}
\subsubsection{综述}
\label{sec-2-2-1}
总共34个冲突,除了if else这类,其他基本是由于括号匹配需要优先移进最后去找失配括号.bison再shift/reduce冲突时优先shift,这证实我们所希望的.
\subsubsection{State 122 conflicts: 1}
\label{sec-2-2-2}
\begin{itemize}
\item if else 的冲突
\item 移进规约冲突的时候默认移进,所以不用具体处理啦.
\item 优先移进正好满足就近匹配的规则,无需处理
\end{itemize}
\subsubsection{State 82 conflicts: 8}
\label{sec-2-2-3}
\begin{itemize}
\item 在MultiBlock(这是我自己定义的)中碰上保留字之类的终结符时出现移进规约冲突
\item 此时优先移进,与if else冲突一样的道理
\item 优先正好满足我们的语言需求,无需处理.
\end{itemize}
\subsubsection{State 66, 67, 68, 69, 70, 71 conflicts: 11}
\label{sec-2-2-4}
\begin{itemize}
\item 这几个冲突原因一样.表达式括号匹配的时候,Exp.error')' 中error的移进规约冲突
\item 优先移进,实际上式默认error是修正过括号之后的表达式.再多一个右括号即触发修正程序
\item 正好满足语言需求,无需处理
\end{itemize}
\subsubsection{State 63 conflicts: 4}
\label{sec-2-2-5}
\begin{itemize}
\item error, ')', '+', '-' 的移进规约冲突
\item 看到规约都是用的57式,即error之后导致需要规约
\item 此时需要移进来查找expr expr这类缺少运算符的错误
\item 优先移进正好满足需求,无需处理.
\end{itemize}
\subsubsection{State 60 conflicts: 7}
\label{sec-2-2-6}
\begin{itemize}
\item 全都是移进27式时的冲突,由blockItem右部可以为空引起
\item 这种可以是0到多个重复的当然直接优先移进就行咯
\end{itemize}
\subsubsection{State 49, 48 conflicts: 2}
\label{sec-2-2-7}
\begin{itemize}
\item 由单目运算符引起, 单目运算符,Exp排起来碰伤 Error的时候出现冲突
\item 因为error可能是要加括号的,我们流到后面再决定,用来匹配那些括号缺失的情况.所以这里先移进
\item 这里优先移进正好是我们需要的.
\end{itemize}
\subsubsection{State 29 conflicts: 1}
\label{sec-2-2-8}
\begin{itemize}
\item 又是和error相关
\item 这里如果又括号,则可能纠正(按照要求纠正三种错误),当然优先移进.
\item 规约是最后迫不得已,三种方式都不能纠正就按要求中止分析.
\end{itemize}

\subsection{碰上的其他问题以及解决方案}
\label{sec-2-3}
\subsubsection{位置}
\label{sec-2-3-1}
parser.y里如果不用位置信息不会在tok.hh里自动生成结构体.尝试的时候一直编译错误找了半天.
\subsubsection{cond无效}
\label{sec-2-3-2}
文法设计不当可能造成了状态机上的孤立节点. 这时候需要小心检查文法. 报了某表达式无效的warning一定要好好检查
\subsubsection{模仿clang输出}
\label{sec-2-3-3}
简单粗暴地把文件先全部读进去存到vector,记好位置之后去掉多余重复的空白字符,用一个尖尖指到要加的位置.
% Emacs 24.5.1 (Org mode 8.2.10)
\end{document}